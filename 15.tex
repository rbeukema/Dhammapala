\newpage
\chapter{Sukhavagga: Happiness}
197. Happy indeed we live, friendly amidst the hostile. Amidst hostile men we dwell free from hatred.

198. Happy indeed we live, friendly amidst the afflicted (by craving). Amidst afflicted men we dwell free from affliction.

199. Happy indeed we live, free from avarice amidst the avaricious. Amidst the avaricious men we dwell free from avarice.

200. Happy indeed we live, we who possess nothing. Feeders on joy we shall be, like the Radiant Gods.

201. Victory begets enmity; the defeated dwell in pain. Happily the peaceful live, discarding both victory and defeat.

202. There is no fire like lust and no crime like hatred. There is no ill like the aggregates (of existence)\footnote{Aggregates (of existence) (khandha): the five groups of factors into which the Buddha analyzes the living being — material form, feeling, perception, mental formations, and consciousness.} and no bliss higher than the peace (of Nibbana).

203. Hunger is the worst disease, conditioned things the worst suffering. Knowing this as it really is, the wise realize Nibbana, the highest bliss.

204. Health is the most precious gain and contentment the greatest wealth. A trustworthy person is the best kinsman, Nibbana the highest bliss.

205. Having savored the taste of solitude and peace (of Nibbana), pain-free and stainless he becomes, drinking deep the taste of the bliss of the Truth.

206. Good is it to see the Noble Ones; to live with them is ever blissful. One will always be happy by not encountering fools.

207. Indeed, he who moves in the company of fools grieves for longing. Association with fools is ever painful, like partnership with an enemy. But association with the wise is happy, like meeting one's own kinsmen.

208. Therefore, follow the Noble One, who is steadfast, wise, learned, dutiful and devout. One should follow only such a man, who is truly good and discerning, even as the moon follows the path of the stars.
