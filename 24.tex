\newpage
\chapter{Tanhavagga: Craving}
334. The craving of one given to heedless living grows like a creeper. Like the monkey seeking fruits in the forest, he leaps from life to life (tasting the fruit of his kamma).

335. Whoever is overcome by this wretched and sticky craving, his sorrows grow like grass after the rains.

336. But whoever overcomes this wretched craving, so difficult to overcome, from him sorrows fall away like water from a lotus leaf.

337. This I say to you: Good luck to all assembled here! Dig up the root of craving, like one in search of the fragrant root of the birana grass. Let not Mara crush you again and again, as a flood crushes a reed.

338. Just as a tree, though cut down, sprouts up again if its roots remain uncut and firm, even so, until the craving that lies dormant is rooted out, suffering springs up again and again.

339. The misguided man in whom the thirty-six currents of craving\footnote{The thirty-six currents of craving: the three cravings — for sensual pleasure, for continued existence, and for annihilation — in relation to each of the twelve bases — the six sense organs, including mind, and their corresponding objects.} strongly rush toward pleasurable objects, is swept away by the flood of his passionate thoughts.

340. Everywhere these currents flow, and the creeper (of craving) sprouts and grows. Seeing that the creeper has sprung up, cut off its root with wisdom.

341. Flowing in (from all objects) and watered by craving, feelings of pleasure arise in beings. Bent on pleasures and seeking enjoyment, these men fall prey to birth and decay.

342. Beset by craving, people run about like an entrapped hare. Held fast by mental fetters, they come to suffering again and again for a long time.

343. Beset by craving, people run about like an entrapped hare. Therefore, one who yearns to be passion-free should destroy his own craving.

344. There is one who, turning away from desire (for household life) takes to the life of the forest (i.e., of a monk). But after being freed from the household, he runs back to it. Behold that man! Though freed, he runs back to that very bondage!\footnote{This verse, in the original, puns with the Pali word vana meaning both "desire" and "forest."}

345-346. That is not a strong fetter, the wise say, which is made of iron, wood or hemp. But the infatuation and longing for jewels and ornaments, children and wives — that, they say, is a far stronger fetter, which pulls one downward and, though seemingly loose, is hard to remove. This, too, the wise cut off. Giving up sensual pleasure, and without any longing, they renounce the world.

347. Those who are lust-infatuated fall back into the swirling current (of samsara) like a spider on its self-spun web. This, too, the wise cut off. Without any longing, they abandon all suffering and renounce the world.

348. Let go of the past, let go of the future, let go of the present, and cross over to the farther shore of existence. With mind wholly liberated, you shall come no more to birth and death.

349. For a person tormented by evil thoughts, who is passion-dominated and given to the pursuit of pleasure, his craving steadily grows. He makes the fetter strong, indeed.

350. He who delights in subduing evil thoughts, who meditates on the impurities and is ever mindful — it is he who will make an end of craving and rend asunder Mara's fetter.

351. He who has reached the goal, is fearless, free from craving, passionless, and has plucked out the thorns of existence — for him this is the last body.

352. He who is free from craving and attachment, is perfect in uncovering the true meaning of the Teaching, and knows the arrangement of the sacred texts in correct sequence — he, indeed, is the bearer of his final body. He is truly called the profoundly wise one, the great man.

353. A victor am I over all, all have I known. Yet unattached am I to all that is conquered and known. Abandoning all, I am freed through the destruction of craving. Having thus directly comprehended all by myself, whom shall I call my teacher?\footnote{This was the Buddha's reply to a wandering ascetic who asked him about his teacher. The Buddha's answer shows that Supreme Enlightenment was his own unique attainment, which he had not learned from anyone else.}

354. The gift of Dhamma excels all gifts; the taste of the Dhamma excels all tastes; the delight in Dhamma excels all delights. The Craving-Freed vanquishes all suffering.

355. Riches ruin only the foolish, not those in quest of the Beyond. By craving for riches the witless man ruins himself as well as others.

356. Weeds are the bane of fields, lust is the bane of mankind. Therefore, what is offered to those free of lust yields abundant fruit.

357. Weeds are the bane of fields, hatred is the bane of mankind. Therefore, what is offered to those free of hatred yields abundant fruit.

358. Weeds are the bane of fields, delusion is the bane of mankind. Therefore, what is offered to those free of delusion yields abundant fruit.

359. Weeds are the bane of fields, desire is the bane of mankind. Therefore, what is offered to those free of desire yields abundant fruit.
