\newpage
\chapter{Brahmanavagga: The Holy Man}
383. Exert yourself, O holy man\footnote{"Holy man" is used as a makeshift rendering for brahmana, intended to reproduce the ambiguity of the Indian word. Originally men of spiritual stature, by the time of the Buddha the brahmans had turned into a privileged priesthood which defined itself by means of birth and lineage rather than by genuine inner sanctity. The Buddha attempted to restore to the word brahmana its original connotation by identifying the true "holy man" as the arahant, who merits the title through his own inward purity and holiness regardless of family lineage. The contrast between the two meanings is highlighted in verses 393 and 396. Those who led a contemplative life dedicated to gaining Arahantship could also be called brahmans, as in verses 383, 389, and 390.}! Cut off the stream (of craving), and discard sense desires. Knowing the destruction of all the conditioned things, become, O holy man, the knower of the Uncreated (Nibbana)!

384. When a holy man has reached the summit of two paths (meditative concentration and insight), he knows the truth and all his fetters fall away.

385. He for whom there is neither this shore nor the other shore, nor yet both, he who is free of cares and is unfettered — him do I call a holy man.\footnote{This shore: the six sense organs; the other shore: their corresponding objects; both: I-ness and my-ness.}

386. He who is meditative, stainless and settled, whose work is done and who is free from cankers, having reached the highest goal — him do I call a holy man.

387. The sun shines by day, the moon shines by night. The warrior shines in armor, the holy man shines in meditation. But the Buddha shines resplendent all day and all night.

388. Because he has discarded evil, he is called a holy man. Because he is serene in conduct, he is called a recluse. And because he has renounced his impurities, he is called a renunciate.

389. One should not strike a holy man, nor should a holy man, when struck, give way to anger. Shame on him who strikes a holy man, and more shame on him who gives way to anger.

390. Nothing is better for a holy man than when he holds his mind back from what is endearing. To the extent the intent to harm wears away, to that extent does suffering subside.

391. He who does no evil in deed, word and thought, who is restrained in these three ways — him do I call a holy man.

392. Just as a brahman priest reveres his sacrificial fire, even so should one devoutly revere the person from whom one has learned the Dhamma taught by the Buddha.

393. Not by matted hair, nor by lineage, nor by birth does one become a holy man. But he in whom truth and righteousness exist — he is pure, he is a holy man.

394. What is the use of your matted hair, O witless man? What of your garment of antelope's hide? Within you is the tangle (of passion); only outwardly do you cleanse yourself.\footnote{In the time of the Buddha, such ascetic practices as wearing matted hair and garments of hides were considered marks of holiness.}

395. The person who wears a robe made of rags, who is lean, with veins showing all over the body, and who meditates alone in the forest — him do I call a holy man.

396. I do not call him a holy man because of his lineage or high-born mother. If he is full of impeding attachments, he is just a supercilious man. But who is free from impediments and clinging — him do I call a holy man.

397. He who, having cut off all fetters, trembles no more, who has overcome all attachments and is emancipated — him do I call a holy man.

398. He who has cut off the thong (of hatred), the band (of craving), and the rope (of false views), together with the appurtenances (latent evil tendencies), he who has removed the crossbar (of ignorance) and is enlightened — him do I call a holy man.

399. He who without resentment endures abuse, beating and punishment; whose power, real might, is patience — him do I call a holy man.

400. He who is free from anger, is devout, virtuous, without craving, self-subdued and bears his final body — him do I call a holy man.

401. Like water on a lotus leaf, or a mustard seed on the point of a needle, he who does not cling to sensual pleasures — him do I call a holy man.

402. He who in this very life realizes for himself the end of suffering, who has laid aside the burden and become emancipated — him do I call a holy man.

403. He who has profound knowledge, who is wise, skilled in discerning the right or wrong path, and has reached the highest goal — him do I call a holy man.

404. He who holds aloof from householders and ascetics alike, and wanders about with no fixed abode and but few wants — him do I call a holy man.

405. He who has renounced violence towards all living beings, weak or strong, who neither kills nor causes others to kill — him do I call a holy man.

406. He who is friendly amidst the hostile, peaceful amidst the violent, and unattached amidst the attached — him do I call a holy man.

407. He whose lust and hatred, pride and hypocrisy have fallen off like a mustard seed from the point of a needle — him do I call a holy man.

408. He who utters gentle, instructive and truthful words, who imprecates none — him do I call a holy man.

409. He who in this world takes nothing that is not given to him, be it long or short, small or big, good or bad — him do I call a holy man.

410. He who wants nothing of either this world or the next, who is desire-free and emancipated — him do I call a holy man.

411. He who has no attachment, who through perfect knowledge is free from doubts and has plunged into the Deathless — him do I call a holy man.

412. He who in this world has transcended the ties of both merit and demerit, who is sorrowless, stainless and pure — him do I call a holy man.

413. He, who, like the moon, is spotless and pure, serene and clear, who has destroyed the delight in existence — him do I call a holy man.

414. He who, having traversed this miry, perilous and delusive round of existence, has crossed over and reached the other shore; who is meditative, calm, free from doubt, and, clinging to nothing, has attained to Nibbana — him do I call a holy man.

415. He who, having abandoned sensual pleasures, has renounced the household life and become a homeless one; has destroyed both sensual desire and continued existence — him do I call a holy man.

416. He who, having abandoned craving, has renounced the household life and become a homeless one, has destroyed both craving and continued existence — him do I call a holy man.

417. He who, casting off human bonds and transcending heavenly ties, is wholly delivered of all bondages — him do I call a holy man.

418. He who, having cast off likes and dislikes, has become tranquil, is rid of the substrata of existence and like a hero has conquered all the worlds — him do I call a holy man.

419. He who in every way knows the death and rebirth of all beings, and is totally detached, blessed and enlightened — him do I call a holy man.

420. He whose track no gods, no angels, no humans trace, the arahant who has destroyed all cankers — him do I call a holy man.

421. He who clings to nothing of the past, present and future, who has no attachment and holds on to nothing — him do I call a holy man.

422. He, the Noble, the Excellent, the Heroic, the Great Sage, the Conqueror, the Passionless, the Pure, the Enlightened one — him do I call a holy man.

423. He who knows his former births, who sees heaven and hell, who has reached the end of births and attained to the perfection of insight, the sage who has reached the summit of spiritual excellence — him do I call a holy man.
