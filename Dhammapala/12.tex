\newpage
\chapter{Attavagga: The Self}

157. If one holds oneself dear, one should diligently watch oneself. Let the wise man keep vigil during any of the three watches of the night.

158. One should first establish oneself in what is proper; then only should one instruct others. Thus the wise man will not be reproached.

159. One should do what one teaches others to do; if one would train others, one should be well controlled oneself. Difficult, indeed, is self-control.

160. One truly is the protector of oneself; who else could the protector be? With oneself fully controlled, one gains a mastery that is hard to gain.

161. The evil a witless man does by himself, born of himself and produced by himself, grinds him as a diamond grinds a hard gem.

162. Just as a single creeper strangles the tree on which it grows, even so, a man who is exceedingly depraved harms himself as only an enemy might wish.

163. Easy to do are things that are bad and harmful to oneself. But exceedingly difficult to do are things that are good and beneficial.

164. Whoever, on account of perverted views, scorns the Teaching of the Perfected Ones, the Noble and Righteous Ones — that fool, like the bamboo, produces fruits only for self destruction.\footnote{Certain reeds of the bamboo family perish immediately after producing fruits.}

165. By oneself is evil done; by oneself is one defiled. By oneself is evil left undone; by oneself is one made pure. Purity and impurity depend on oneself; no one can purify another.

166. Let one not neglect one's own welfare for the sake of another, however great. Clearly understanding one's own welfare, let one be intent upon the good.
