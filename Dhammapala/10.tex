\newpage
\chapter{Dandavagga: Violence}

129. All tremble at violence; all fear death. Putting oneself in the place of another, one should not kill nor cause another to kill.

130. All tremble at violence; life is dear to all. Putting oneself in the place of another, one should not kill nor cause another to kill.

131. One who, while himself seeking happiness, oppresses with violence other beings who also desire happiness, will not attain happiness hereafter.

132. One who, while himself seeking happiness, does not oppress with violence other beings who also desire happiness, will find happiness hereafter.

133. Speak not harshly to anyone, for those thus spoken to might retort. Indeed, angry speech hurts, and retaliation may overtake you.

134. If, like a broken gong, you silence yourself, you have approached Nibbana, for vindictiveness is no longer in you.

135. Just as a cowherd drives the cattle to pasture with a staff, so do old age and death drive the life force of beings (from existence to existence).

136. When the fool commits evil deeds, he does not realize (their evil nature). The witless man is tormented by his own deeds, like one burnt by fire.

137. He who inflicts violence on those who are unarmed, and offends those who are inoffensive, will soon come upon one of these ten states:

138-140 Sharp pain, or disaster, bodily injury, serious illness, or derangement of mind, trouble from the government, or grave charges, loss of relatives, or loss of wealth, or houses destroyed by ravaging fire; upon dissolution of the body that ignorant man is born in hell.

141. Neither going about naked, nor matted locks, nor filth, nor fasting, nor lying on the ground, nor smearing oneself with ashes and dust, nor sitting on the heels (in penance) can purify a mortal who has not overcome doubt.

142. Even though he be well-attired, yet if he is poised, calm, controlled and established in the holy life, having set aside violence towards all beings — he, truly, is a holy man, a renunciate, a monk.

143. Only rarely is there a man in this world who, restrained by modesty, avoids reproach, as a thoroughbred horse avoids the whip.

144. Like a thoroughbred horse touched by the whip, be strenuous, be filled with spiritual yearning. By faith and moral purity, by effort and meditation, by investigation of the truth, by being rich in knowledge and virtue, and by being mindful, destroy this unlimited suffering.

145. Irrigators regulate the waters, fletchers straighten arrow shafts, carpenters shape wood, and the good control themselves.
