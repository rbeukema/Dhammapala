\newpage
\chapter{Dhammatthavagga: The Just}
256. Not by passing arbitrary judgments does a man become just; a wise man is he who investigates both right and wrong.

257. He who does not judge others arbitrarily, but passes judgment impartially according to the truth, that sagacious man is a guardian of law and is called just.

258. One is not wise because one speaks much. He who is peaceable, friendly and fearless is called wise.

259. A man is not versed in Dhamma because he speaks much. He who, after hearing a little Dhamma, realizes its truth directly and is not heedless of it, is truly versed in the Dhamma.

260. A monk is not an elder because his head is gray. He is but ripe in age, and he is called one grown old in vain.

261. One in whom there is truthfulness, virtue, inoffensiveness, restraint and self-mastery, who is free from defilements and is wise — he is truly called an Elder.

262. Not by mere eloquence nor by beauty of form does a man become accomplished, if he is jealous, selfish and deceitful.

263. But he in whom these are wholly destroyed, uprooted and extinct, and who has cast out hatred — that wise man is truly accomplished.

264. Not by shaven head does a man who is indisciplined and untruthful become a monk. How can he who is full of desire and greed be a monk?

265. He who wholly subdues evil both small and great is called a monk, because he has overcome all evil.

266. He is not a monk just because he lives on others' alms. Not by adopting outward form does one become a true monk.

267. Whoever here (in the Dispensation) lives a holy life, transcending both merit and demerit, and walks with understanding in this world — he is truly called a monk.

268. Not by observing silence does one become a sage, if he be foolish and ignorant. But that man is wise who, as if holding a balance-scale accepts only the good.

269. The sage (thus) rejecting the evil, is truly a sage. Since he comprehends both (present and future) worlds, he is called a sage.

270. He is not noble who injures living beings. He is called noble because he is harmless towards all living beings.

271-272. Not by rules and observances, not even by much learning, nor by gain of absorption, nor by a life of seclusion, nor by thinking, "I enjoy the bliss of renunciation, which is not experienced by the worldling" should you, O monks, rest content, until the utter destruction of cankers (Arahantship) is reached.
