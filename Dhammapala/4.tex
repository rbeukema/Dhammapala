\newpage
\chapter{Pupphavagga: Flowers}

44. Who shall overcome this earth, this realm of Yama and this sphere of men and gods? Who shall bring to perfection the well-taught path of wisdom as an expert garland-maker would his floral design?

45. A Striver-on-the-Path\footnote{The Striver-on-the-Path (sekha): one who has achieved any of the first three stages of supramundane attainment: a stream-enterer, once-returner, or non-returner.} shall overcome this earth, this realm of Yama and this sphere of men and gods. The striver-on-the-path shall bring to perfection the well-taught path of wisdom, as an expert garland-maker would his floral design.

46. Realizing that this body is like froth, penetrating its mirage-like nature, and plucking out Mara's flower-tipped arrows of sensuality, go beyond sight of the King of Death!

47. As a mighty flood sweeps away the sleeping village, so death carries away the person of distracted mind who only plucks the flowers (of pleasure).

48. The Destroyer brings under his sway the person of distracted mind who, insatiate in sense desires, only plucks the flowers (of pleasure).

49. As a bee gathers honey from the flower without injuring its color or fragrance, even so the sage goes on his alms-round in the village.\footnote{The "sage in the village" is the Buddhist monk who receives his food by going silently from door to door with his alms bowls, accepting whatever is offered.}

50. Let none find fault with others; let none see the omissions and commissions of others. But let one see one's own acts, done and undone.

51. Like a beautiful flower full of color but without fragrance, even so, fruitless are the fair words of one who does not practice them.

52. Like a beautiful flower full of color and also fragrant, even so, fruitful are the fair words of one who practices them.

53. As from a great heap of flowers many garlands can be made, even so should many good deeds be done by one born a mortal.

54. Not the sweet smell of flowers, not even the fragrance of sandal, tagara\footnote{Tagara: a fragrant powder obtained from a particular kind of shrub.}, or jasmine blows against the wind. But the fragrance of the virtuous blows against the wind. Truly the virtuous man pervades all directions with the fragrance of his virtue.

55. Of all the fragrances — sandal, tagara, blue lotus and jasmine — the fragrance of virtue is the sweetest.

56. Faint is the fragrance of tagara and sandal, but excellent is the fragrance of the virtuous, wafting even amongst the gods.

57. Mara never finds the path of the truly virtuous, who abide in heedfulness and are freed by perfect knowledge.

58. Upon a heap of rubbish in the road-side ditch blooms a lotus, fragrant and pleasing.

59. Even so, on the rubbish heap of blinded mortals the disciple of the Supremely Enlightened One shines resplendent in wisdom.
