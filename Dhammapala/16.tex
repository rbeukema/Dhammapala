\newpage
\chapter{Piyavagga: Affection}
209. Giving himself to things to be shunned and not exerting where exertion is needed, a seeker after pleasures, having given up his true welfare, envies those intent upon theirs.

210. Seek no intimacy with the beloved and also not with the unloved, for not to see the beloved and to see the unloved, both are painful.

211. Therefore hold nothing dear, for separation from the dear is painful. There are no bonds for those who have nothing beloved or unloved.

212. From endearment springs grief, from endearment springs fear. For one who is wholly free from endearment there is no grief, whence then fear?

213. From affection springs grief, from affection springs fear. For one who is wholly free from affection there is no grief, whence then fear?

214. From attachment springs grief, from attachment springs fear. For one who is wholly free from attachment there is no grief, whence then fear?

215. From lust springs grief, from lust springs fear. For one who is wholly free from craving there is no grief; whence then fear?

216. From craving springs grief, from craving springs fear. For one who is wholly free from craving there is no grief; whence then fear?

217. People hold dear him who embodies virtue and insight, who is principled, has realized the truth, and who himself does what he ought to be doing.

218. One who is intent upon the Ineffable (Nibbana), dwells with mind inspired (by supramundane wisdom), and is no more bound by sense pleasures — such a man is called "One Bound Upstream\footnote{One Bound Upstream: a non-returner (anagami).}."

219. When, after a long absence, a man safely returns from afar, his relatives, friends and well-wishers welcome him home on arrival.

220. As kinsmen welcome a dear one on arrival, even so his own good deeds will welcome the doer of good who has gone from this world to the next.
