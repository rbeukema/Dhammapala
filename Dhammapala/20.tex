\newpage
\chapter{Maggavagga: The Path}
273. Of all the paths the Eightfold Path is the best; of all the truths the Four Noble Truths are the best; of all things passionlessness is the best: of men the Seeing One (the Buddha) is the best.

274. This is the only path; there is none other for the purification of insight. Tread this path, and you will bewilder Mara.

275. Walking upon this path you will make an end of suffering. Having discovered how to pull out the thorn of lust, I make known the path.

276. You yourselves must strive; the Buddhas only point the way. Those meditative ones who tread the path are released from the bonds of Mara.

277. "All conditioned things are impermanent" — when one sees this with wisdom, one turns away from suffering. This is the path to purification.

278. "All conditioned things are unsatisfactory" — when one sees this with wisdom, one turns away from suffering. This is the path to purification.

279. "All things are not-self" — when one sees this with wisdom, one turns away from suffering. This is the path to purification.

280. The idler who does not exert himself when he should, who though young and strong is full of sloth, with a mind full of vain thoughts — such an indolent man does not find the path to wisdom.

281. Let a man be watchful of speech, well controlled in mind, and not commit evil in bodily action. Let him purify these three courses of action, and win the path made known by the Great Sage.

282. Wisdom springs from meditation; without meditation wisdom wanes. Having known these two paths of progress and decline, let a man so conduct himself that his wisdom may increase.

283. Cut down the forest (lust), but not the tree; from the forest springs fear. Having cut down the forest and the underbrush (desire), be passionless, O monks!\footnote{The meaning of this injunction is: "Cut down the forest of lust, but do not mortify the body."}

284. For so long as the underbrush of desire, even the most subtle, of a man towards a woman is not cut down, his mind is in bondage, like the sucking calf to its mother.

285. Cut off your affection in the manner of a man who plucks with his hand an autumn lotus. Cultivate only the path to peace, Nibbana, as made known by the Exalted One.

286. "Here shall I live during the rains, here in winter and summer" — thus thinks the fool. He does not realize the danger (that death might intervene).

287. As a great flood carries away a sleeping village, so death seizes and carries away the man with a clinging mind, doting on his children and cattle.

288. For him who is assailed by death there is no protection by kinsmen. None there are to save him — no sons, nor father, nor relatives.

289. Realizing this fact, let the wise man, restrained by morality, hasten to clear the path leading to Nibbana.
