\newpage
\chapter{Jaravagga: Old Age}

146. When this world is ever ablaze, why this laughter, why this jubilation? Shrouded in darkness, will you not see the light?

147. Behold this body — a painted image, a mass of heaped up sores, infirm, full of hankering — of which nothing is lasting or stable!

148. Fully worn out is this body, a nest of disease, and fragile. This foul mass breaks up, for death is the end of life.

149. These dove-colored bones are like gourds that lie scattered about in autumn. Having seen them, how can one seek delight?

150. This city (body) is built of bones, plastered with flesh and blood; within are decay and death, pride and jealousy.

151. Even gorgeous royal chariots wear out, and indeed this body too wears out. But the Dhamma of the Good does not age; thus the Good make it known to the good.

152. The man of little learning grows old like a bull. He grows only in bulk, but, his wisdom does not grow.

153. Through many a birth in samsara have I wandered in vain, seeking the builder of this house (of life). Repeated birth is indeed suffering!

154. O house-builder, you are seen! You will not build this house again. For your rafters are broken and your ridgepole shattered. My mind has reached the Unconditioned; I have attained the destruction of craving.\footnote{vv. 153-154 - According to the commentary, these verses are the Buddha's "Song of Victory," his first utterance after his Enlightenment. The house is individualized existence in samsara, the house-builder craving, the rafters the passions and the ridge-pole ignorance.}

155. Those who in youth have not led the holy life, or have failed to acquire wealth, languish like old cranes in the pond without fish.

156. Those who in youth have not lead the holy life, or have failed to acquire wealth, lie sighing over the past, like worn out arrows (shot from) a bow.
