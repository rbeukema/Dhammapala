\newpage
\chapter{Malavagga: Impurity}
235. Like a withered leaf are you now; death's messengers await you. You stand on the eve of your departure, yet you have made no provision for your journey!

236. Make an island for yourself! Strive hard and become wise! Rid of impurities and cleansed of stain, you shall enter the celestial abode of the Noble Ones.

237. Your life has come to an end now; You are setting forth into the presence of Yama, the king of death. No resting place is there for you on the way, yet you have made no provision for the journey!

238. Make an island unto yourself! Strive hard and become wise! Rid of impurities and cleansed of stain, you shall not come again to birth and decay.

239. One by one, little by little, moment by moment, a wise man should remove his own impurities, as a smith removes his dross from silver.

240. Just as rust arising from iron eats away the base from which it arises, even so, their own deeds lead transgressors to states of woe.

241. Non-repetition is the bane of scriptures; neglect is the bane of a home; slovenliness is the bane of personal appearance, and heedlessness is the bane of a guard.

242. Unchastity is the taint in a woman; niggardliness is the taint in a giver. Taints, indeed, are all evil things, both in this world and the next.

243. A worse taint than these is ignorance, the worst of all taints. Destroy this one taint and become taintless, O monks!

244. Easy is life for the shameless one who is impudent as a crow, is backbiting and forward, arrogant and corrupt.

245. Difficult is life for the modest one who always seeks purity, is detached and unassuming, clean in life, and discerning.

246-247. One who destroys life, utters lies, takes what is not given, goes to another man's wife, and is addicted to intoxicating drinks — such a man digs up his own root even in this world.

248. Know this, O good man: evil things are difficult to control. Let not greed and wickedness drag you to protracted misery.

249. People give according to their faith or regard. If one becomes discontented with the food and drink given by others, one does not attain meditative absorption, either by day or by night.

250. But he in who this (discontent) is fully destroyed, uprooted and extinct, he attains absorption, both by day and by night.

251. There is no fire like lust; there is no grip like hatred; there is no net like delusion; there is no river like craving.

252. Easily seen is the fault of others, but one's own fault is difficult to see. Like chaff one winnows another's faults, but hides one's own, even as a crafty fowler hides behind sham branches.

253. He who seeks another's faults, who is ever censorious — his cankers grow. He is far from destruction of the cankers.

254. There is no track in the sky, and no recluse\footnote{vv. 254-255 - Recluse (samana): here used in the special sense of those who have reached the four supramundane stages.} outside (the Buddha's dispensation). Mankind delights in worldliness, but the Buddhas are free from worldliness.

255. There is no track in the sky, and no recluse outside (the Buddha's dispensation). There are no conditioned things that are eternal, and no instability in the Buddhas.
