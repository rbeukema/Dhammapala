\newpage
\chapter{Papavagga: Evil}

116. Hasten to do good; restrain your mind from evil. He who is slow in doing good, his mind delights in evil.

117. Should a person commit evil, let him not do it again and again. Let him not find pleasure therein, for painful is the accumulation of evil.

118. Should a person do good, let him do it again and again. Let him find pleasure therein, for blissful is the accumulation of good.

119. It may be well with the evil-doer as long as the evil ripens not. But when it does ripen, then the evil-doer sees (the painful results of) his evil deeds.

120. It may be ill with the doer of good as long as the good ripens not. But when it does ripen, then the doer of good sees (the pleasant results of) his good deeds.

121. Think not lightly of evil, saying, "It will not come to me." Drop by drop is the water pot filled. Likewise, the fool, gathering it little by little, fills himself with evil.

122. Think not lightly of good, saying, "It will not come to me." Drop by drop is the water pot filled. Likewise, the wise man, gathering it little by little, fills himself with good.

123. Just as a trader with a small escort and great wealth would avoid a perilous route, or just as one desiring to live avoids poison, even so should one shun evil.

124. If on the hand there is no wound, one may carry even poison in it. Poison does not affect one who is free from wounds. For him who does no evil, there is no ill.

125. Like fine dust thrown against the wind, evil falls back upon that fool who offends an inoffensive, pure and guiltless man.

126. Some are born in the womb; the wicked are born in hell; the devout go to heaven; the stainless pass into Nibbana.

127. Neither in the sky nor in mid-ocean, nor by entering into mountain clefts, nowhere in the world is there a place where one may escape from the results of evil deeds.

128. Neither in the sky nor in mid-ocean, nor by entering into mountain clefts, nowhere in the world is there a place where one will not be overcome by death.
