\chapter{Appamadavagga: Heedfulness}
\epigraphhead[0]{\epigraph{\textit{``Do not dwell in the past, do not dream of the future, concentrate the mind on the present moment."}}{Gautama Buddha}}

21. Heedfulness is the path to the Deathless\footnote{The Deathless (amata): Nibbana, so called because those who attain it are free from the cycle of repeated birth and death.}. Heedlessness is the path to death. The heedful die not. The heedless are as if dead already.

22. Clearly understanding this excellence of heedfulness, the wise exult therein and enjoy the resort of the Noble Ones\footnote{The Noble Ones (ariya): those who have reached any of the four stages of supramundane attainment leading irreversibly to Nibbana.}.

23. The wise ones, ever meditative and steadfastly persevering, alone experience Nibbana, the incomparable freedom from bondage.

24. Ever grows the glory of him who is energetic, mindful and pure in conduct, discerning and self-controlled, righteous and heedful.

25. By effort and heedfulness, discipline and self-mastery, let the wise one make for himself an island which no flood can overwhelm.

26. The foolish and ignorant indulge in heedlessness, but the wise one keeps his heedfulness as his best treasure.

27. Do not give way to heedlessness. Do not indulge in sensual pleasures. Only the heedful and meditative attain great happiness.

28. Just as one upon the summit of a mountain beholds the groundlings, even so when the wise man casts away heedlessness by heedfulness and ascends the high tower of wisdom, this sorrowless sage beholds the sorrowing and foolish multitude.

29. Heedful among the heedless, wide-awake among the sleepy, the wise man advances like a swift horse leaving behind a weak jade.

30. By Heedfulness did Indra\footnote{Indra: the ruler of the gods in ancient Indian mythology.} become the overlord of the gods. Heedfulness is ever praised, and heedlessness ever despised.

31. The monk who delights in heedfulness and looks with fear at heedlessness advances like fire, burning all fetters, small and large.

32. The monk who delights in heedfulness and looks with fear at heedlessness will not fall. He is close to Nibbana.
