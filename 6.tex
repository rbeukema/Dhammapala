\newpage
\chapter{Panditavagga: The Wise}

76. Should one find a man who points out faults and who reproves, let him follow such a wise and sagacious person as one would a guide to hidden treasure. It is always better, and never worse, to cultivate such an association.

77. Let him admonish, instruct and shield one from wrong; he, indeed, is dear to the good and detestable to the evil.

78. Do not associate with evil companions; do not seek the fellowship of the vile. Associate with the good friends; seek the fellowship of noble men.

79. He who drinks deep the Dhamma lives happily with a tranquil mind. The wise man ever delights in the Dhamma made known by the Noble One (the Buddha).

80. Irrigators regulate the rivers; fletchers straighten the arrow shaft; carpenters shape the wood; the wise control themselves.

81. Just as a solid rock is not shaken by the storm, even so the wise are not affected by praise or blame.

82. On hearing the Teachings, the wise become perfectly purified, like a lake deep, clear and still.

83. The good renounce (attachment for) everything. The virtuous do not prattle with a yearning for pleasures. The wise show no elation or depression when touched by happiness or sorrow.

84. He is indeed virtuous, wise, and righteous who neither for his own sake nor for the sake of another (does any wrong), who does not crave for sons, wealth, or kingdom, and does not desire success by unjust means.

85. Few among men are those who cross to the farther shore. The rest, the bulk of men, only run up and down the hither bank.

86. But those who act according to the perfectly taught Dhamma will cross the realm of Death, so difficult to cross.

87-88. Abandoning the dark way, let the wise man cultivate the bright path. Having gone from home to homelessness, let him yearn for that delight in detachment, so difficult to enjoy. Giving up sensual pleasures, with no attachment, let the wise man cleanse himself of defilements of the mind.

89. Those whose minds have reached full excellence in the factors of enlightenment, who, having renounced acquisitiveness, rejoice in not clinging to things — rid of cankers, glowing with wisdom, they have attained Nibbana in this very life. \footnote{This verse describes the arahant, dealt with more fully in the following chapter. The "cankers" (asava) are the four basic defilements of sensual desire, desire for continued existence, false views and ignorance.}
