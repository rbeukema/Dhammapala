\newpage
\chapter{Introduction}
	From ancient times to the present, the Dhammapada has been regarded as the most succinct expression of the Buddha's teaching found in the Pali canon and the chief spiritual testament of early Buddhism. In the countries following Theravada Buddhism, such as Sri Lanka, Burma and Thailand, the influence of the Dhammapada is ubiquitous. It is an ever-fecund source of themes for sermons and discussions, a guidebook for resolving the countless problems of everyday life, a primer for the instruction of novices in the monasteries. Even the experienced contemplative, withdrawn to forest hermitage or mountainside cave for a life of meditation, can be expected to count a copy of the book among his few material possessions. Yet the admiration the Dhammapada has elicited has not been confined to avowed followers of Buddhism. Wherever it has become known its moral earnestness, realistic understanding of human life, aphoristic wisdom and stirring message of a way to freedom from suffering have won for it the devotion and veneration of those responsive to the good and the true.
	The expounder of the verses that comprise the Dhammapada is the Indian sage called the Buddha, an honorific title meaning "the Enlightened One" or "the Awakened One." The story of this venerable personage has often been overlaid with literary embellishment and the admixture of legend, but the historical essentials of his life are simple and clear. He was born in the sixth century B.C., the son of a king ruling over a small state in the Himalayan foothills, in what is now Nepal. His given name was Siddhattha and his family name Gotama (Sanskrit: Siddhartha Gautama) . Raised in luxury, groomed by his father to be the heir to the throne, in his early manhood he went through a deeply disturbing encounter with the sufferings of life, as a result of which he lost all interest in the pleasures and privileges of rulership. One night, in his twenty-ninth year, he fled the royal city and entered the forest to live as an ascetic, resolved to find a way to deliverance from suffering. For six years he experimented with different systems of meditation and subjected himself to severe austerities, but found that these practices did not bring him any closer to his goal. Finally, in his thirty-fifth year, while sitting in deep meditation beneath a tree at Gaya, he attained Supreme Enlightenment and became, in the proper sense of the title, the Buddha, the Enlightened One. Thereafter, for forty-five years, he traveled throughout northern India, proclaiming the truths he had discovered and founding an order of monks and nuns to carry on his message. At the age of eighty, after a long and fruitful life, he passed away peacefully in the small town of Kusinara, surrounded by a large number of disciples.
	To his followers, the Buddha is neither a god, a divine incarnation, or a prophet bearing a message of divine revelation, but a human being who by his own striving and intelligence has reached the highest spiritual attainment of which man is capable — perfect wisdom, full enlightenment, complete purification of mind. His function in relation to humanity is that of a teacher — a world teacher who, out of compassion, points out to others the way to Nibbana (Sanskrit: Nirvana), final release from suffering. His teaching, known as the Dhamma, offers a body of instructions explaining the true nature of existence and showing the path that leads to liberation. Free from all dogmas and inscrutable claims to authority, the Dhamma is founded solidly upon the bedrock of the Buddha's own clear comprehension of reality, and it leads the one who practices it to that same understanding — the knowledge which extricates the roots of suffering.
	The title "Dhammapada" which the ancient compilers of the Buddhist scriptures attached to our anthology means portions, aspects, or sections of Dhamma. The work has been given this title because, in its twenty-six chapters, it spans the multiple aspects of the Buddha's teaching, offering a variety of standpoints from which to gain a glimpse into its heart. Whereas the longer discourses of the Buddha contained in the prose sections of the Canon usually proceed methodically, unfolding according to the sequential structure of the doctrine, the Dhammapada lacks such a systematic arrangement. The work is simply a collection of inspirational or pedagogical verses on the fundamentals of the Dhamma, to be used as a basis for personal edification and instruction. In any given chapter several successive verses may have been spoken by the Buddha on a single occasion, and thus among themselves will exhibit a meaningful development or a set of variations on a theme. But by and large, the logic behind the grouping together of verses into a chapter is merely the concern with a common topic. The twenty-six chapter headings thus function as a kind of rubric for classifying the diverse poetic utterances of the Master, and the reason behind the inclusion of any given verse in a particular chapter is its mention of the subject indicated in the chapter's heading . In some cases (Chapters 4 and 23) this may be a metaphorical symbol rather than a point of doctrine. There also seems to be no intentional design in the order of the chapters themselves, though at certain points a loose thread of development can be discerned.
	The teachings of the Buddha, viewed in their completeness, all link together into a single perfectly coherent system of thought and practice which gains its unity from its final goal, the attainment of deliverance from suffering. But the teachings inevitably emerge from the human condition as their matrix and starting point, and thus must be expressed in such a way as to reach human beings standing at different levels of spiritual development, with their highly diverse problems, ends, and concerns and with their very different capacities for understanding. Thence, just as water, though one in essence, assumes different shapes due to the vessels into which it is poured, so the Dhamma of liberation takes on different forms in response to the needs of the beings to be taught. This diversity, evident enough already in the prose discourses, becomes even more conspicuous in the highly condensed, spontaneous and intuitively charged medium of verse used in the Dhammapada. The intensified power of delivery can result in apparent inconsistencies which may perplex the unwary. For example, in many verses the Buddha commends certain practices on the grounds that they lead to a heavenly birth, but in others he discourages disciples from aspiring for heaven and extols the one who takes no delight in celestial pleasures (187, 417) [Unless chapter numbers are indicated, all figures enclosed in parenthesis refer to verse numbers of the Dhammapada.]
	Often he enjoins works of merit, yet elsewhere he praises the one who has gone beyond both merit and demerit (39, 412). Without a grasp of the underlying structure of the Dhamma, such statements viewed side by side will appear incompatible and may even elicit the judgment that the teaching is self-contradictory.
	The key to resolving these apparent discrepancies is the recognition that the Dhamma assumes its formulation from the needs of the diverse persons to whom it is addressed, as well as from the diversity of needs that may co-exist even in a single individual. To make sense of the various utterances found in the Dhammapada, we will suggest a schematism of four levels to be used for ascertaining the intention behind any particular verse found in the work, and thus for understanding its proper place in the total systematic vision of the Dhamma. This fourfold schematism develops out of an ancient interpretive maxim which holds that the Buddha's teaching is designed to meet three primary aims: human welfare here and now, a favorable rebirth in the next life, and the attainment of the ultimate good. The four levels are arrived at by distinguishing the last aim into two stages: path and fruit.
\begin{enumerate}[i]
\item The first level is the concern with establishing well-being and happiness in the immediately visible sphere of concrete human relations. The aim at this level is to show man the way to live at peace with himself and his fellow men, to fulfill his family and social responsibilities, and to restrain the bitterness, conflict and violence which infect human relationships and bring such immense suffering to the individual, society, and the world as a whole. The guidelines appropriate to this level are largely identical with the basic ethical injunctions proposed by most of the great world religions, but in the Buddhist teaching they are freed from theistic moorings and grounded upon two directly verifiable foundations: concern for one's own integrity and long-range happiness and concern for the welfare of those whom one's actions may affect (129-132). The most general counsel the Dhammapada gives is to avoid all evil, to cultivate good and to cleanse one's mind (183). But to dispel any doubts the disciple might entertain as to what he should avoid and what he should cultivate, other verses provide more specific directives. One should avoid irritability in deed, word and thought and exercise self-control (231-234). One should adhere to the five precepts, the fundamental moral code of Buddhism, which teach abstinence from destroying life, from stealing, from committing adultery, from speaking lies and from taking intoxicants; one who violates these five training rules "digs up his own root even in this very world" (246-247). The disciple should treat all beings with kindness and compassion, live honestly and righteously, control his sensual desires, speak the truth and live a sober upright life, diligently fulfilling his duties, such as service to parents, to his immediate family and to those recluses and brahmans who depend on the laity for their maintenance (332-333).

A large number of verses pertaining to this first level are concerned with the resolution of conflict and hostility. Quarrels are to be avoided by patience and forgiveness, for responding to hatred by further hatred only maintains the cycle of vengeance and retaliation. The true conquest of hatred is achieved by non-hatred, by forbearance, by love (4-6). One should not respond to bitter speech but maintain silence (134). One should not yield to anger but control it as a driver controls a chariot (222). Instead of keeping watch for the faults of others, the disciple is admonished to examine his own faults, and to make a continual effort to remove his impurities just as a silversmith purifies silver (50, 239). Even if he has committed evil in the past, there is no need for dejection or despair; for a man's ways can be radically changed, and one who abandons the evil for the good illuminates this world like the moon freed from clouds (173).

The sterling qualities distinguishing the man of virtue are generosity, truthfulness, patience, and compassion (223). By developing and mastering these qualities within himself, a man lives at harmony with his own conscience and at peace with his fellow beings. The scent of virtue, the Buddha declares, is sweeter than the scent of all flowers and perfumes (55-56). The good man, like the Himalaya mountains, shines from afar, and wherever he goes he is loved and respected (303-304).

\item In its second level of teaching, the Dhammapada shows that morality does not exhaust its significance in its contribution to human felicity here and now, but exercises a far more critical influence in molding personal destiny. This level begins with the recognition that, to reflective thought, the human situation demands a more satisfactory context for ethics than mere appeals to altruism can provide. On the one hand our innate sense of moral justice requires that goodness be recompensed with happiness and evil with suffering; on the other our typical experience shows us virtuous people beset with hardships and afflictions and thoroughly bad people riding the waves of fortune (119-120). Moral intuition tells us that if there is any long-range value to righteousness, the imbalance must somehow be redressed. The visible order does not yield an evident solution, but the Buddha's teaching reveals the factor needed to vindicate our cry for moral justice in an impersonal universal law which reigns over all sentient existence. This is the law of kamma (Sanskrit: karma), of action and its fruit, which ensures that morally determinate action does not disappear into nothingness but eventually meets its due retribution, the good with happiness, the bad with suffering.

In the popular understanding kamma is sometimes identified with fate, but this is a total misconception utterly inapplicable to the Buddhist doctrine. Kamma means volitional action, action springing from intention, which may manifest itself outwardly as bodily deeds or speech, or remain internally as unexpressed thoughts, desires and emotions. The Buddha distinguishes kamma into two primary ethical types: unwholesome kamma, action rooted in mental states of greed, hatred and delusion; and wholesome kamma, action rooted in mental states of generosity or detachment, goodwill and understanding. The willed actions a person performs in the course of his life may fade from memory without a trace, but once performed they leave subtle imprints on the mind, seeds with the potential to come to fruition in the future when they meet conditions conducive to their ripening.

The objective field in which the seeds of kamma ripen is the process of rebirths called samsara. In the Buddha's teaching, life is not viewed as an isolated occurrence beginning spontaneously with birth and ending in utter annihilation at death. Each single life span is seen, rather, as part of an individualized series of lives having no discoverable beginning in time and continuing on as long as the desire for existence stands intact. Rebirth can take place in various realms. There are not only the familiar realms of human beings and animals, but ranged above we meet heavenly worlds of greater happiness, beauty and power, and ranged below infernal worlds of extreme suffering.

The cause for rebirth into these various realms the Buddha locates in kamma, our own willed actions. In its primary role, kamma determines the sphere into which rebirth takes place, wholesome actions bringing rebirth in higher forms, unwholesome actions rebirth in lower forms. After yielding rebirth, kamma continues to operate, governing the endowments and circumstances of the individual within his given form of existence. Thus, within the human world, previous stores of wholesome kamma will issue in long life, health, wealth, beauty and success; stores of unwholesome kamma in short life, illness, poverty, ugliness and failure.

Prescriptively, the second level of teaching found in the Dhammapada is the practical corollary to this recognition of the law of kamma, put forth to show human beings, who naturally desire happiness and freedom from sorrow, the effective means to achieve their objectives. The content of this teaching itself does not differ from that presented at the first level; it is the same set of ethical injunctions for abstaining from evil and for cultivating the good. The difference lies in the perspective from which the injunctions are issued and the aim for the sake of which they are to be taken up. The principles of morality are shown now in their broader cosmic connections, as tied to an invisible but all-embracing law which binds together all life and holds sway over the repeated rotations of the cycle of birth and death. The observance of morality is justified, despite its difficulties and apparent failures, by the fact that it is in harmony with that law, that through the efficacy of kamma, our willed actions become the chief determinant of our destiny both in this life and in future states of becoming. To follow the ethical law leads upwards — to inner development, to higher rebirths and to richer experiences of happiness and joy. To violate the law, to act in the grip of selfishness and hate, leads downwards — to inner deterioration, to suffering and to rebirth in the worlds of misery. This theme is announced already by the pair of verses which opens the Dhammapada, and reappears in diverse formulations throughout the work (see, e.g., 15-18, 117-122, 127, 132-133, Chapter 22).

\item The ethical counsel based on the desire for higher rebirths and happiness in future lives is not the final teaching of the Buddha, and thus cannot provide the decisive program of personal training commended by the Dhammapada. In its own sphere of application, it is perfectly valid as a preparatory or provisional teaching for those whose spiritual faculties are not yet ripe but still require further maturation over a succession of lives. A deeper, more searching examination, however, reveals that all states of existence in samsara, even the loftiest celestial abodes, are lacking in genuine worth; for they are all inherently impermanent, without any lasting substance, and thus, for those who cling to them, potential bases for suffering. The disciple of mature faculties, sufficiently prepared by previous experience for the Buddha's distinctive exposition of the Dhamma, does not long even for rebirth among the gods. Having understood the intrinsic inadequacy of all conditioned things, his focal aspiration is only for deliverance from the ever-repeating round of births. This is the ultimate goal to which the Buddha points, as the immediate aim for those of developed faculties and also as the long-term ideal for those in need of further development: Nibbana, the Deathless, the unconditioned state where there is no more birth, aging and death, and no more suffering.

The third level of teaching found in the Dhammapada sets forth the theoretical framework and practical discipline emerging out of the aspiration for final deliverance. The theoretical framework is provided by the teaching of the Four Noble Truths (190-192, 273), which the Buddha had proclaimed already in his first sermon and upon which he placed so much stress in his many discourses that all schools of Buddhism have appropriated them as their common foundation. The four truths all center around the fact of suffering (dukkha), understood not as mere experienced pain and sorrow, but as the pervasive unsatisfactoriness of everything conditioned (202-203). The first truth details the various forms of suffering — birth, old age, sickness and death, the misery of unpleasant encounters and painful separations, the suffering of not obtaining what one wants. It culminates in the declaration that all constituent phenomena of body and mind, "the aggregates of existence" (khandha), being impermanent and substanceless, are intrinsically unsatisfactory. The second truth points out that the cause of suffering is craving (tanha), the desire for pleasure and existence which drives us through the round of rebirths, bringing in its trail sorrow, anxiety, and despair (212-216, Chapter 24). The third truth declares that the destruction of craving issues in release from suffering, and the fourth prescribes the means to gain release, the Noble Eightfold Path: right understanding, right thought, right speech, right action, right livelihood, right effort, right mindfulness, and right concentration (Chapter 20).

If, at this third level, the doctrinal emphasis shifts from the principles of kamma and rebirth to the Four Noble Truths, a corresponding shift in emphasis takes place in the practical sphere as well. The stress now no longer falls on the observation of basic morality and the cultivation of wholesome attitudes as a means to higher rebirths. Instead it falls on the integral development of the Noble Eightfold Path as the means to uproot the craving that nurtures the process of rebirth itself. For practical purposes the eight factors of the path are arranged into three major groups which reveal more clearly the developmental structure of the training: moral discipline (including right speech, right action and right livelihood), concentration (including right effort, right mindfulness and right concentration), and wisdom (including right understanding and right thought). By the training in morality, the coarsest forms of the mental defilements, those erupting as unwholesome deeds and words, are checked and kept under control. By the training in concentration the mind is made calm, pure and unified, purged of the currents of distractive thoughts. By the training in wisdom the concentrated beam of attention is focused upon the constituent factors of mind and body to investigate and contemplate their salient characteristics. This wisdom, gradually ripened, climaxes in the understanding that brings complete purification and deliverance of mind.

In principle, the practice of the path in all three stages is feasible for people in any walk of life. The Buddha taught it to laypeople as well as to monks, and many of his lay followers reached high stages of attainment. However, application to the development of the path becomes most fruitful for those who have relinquished all other concerns in order to devote themselves wholeheartedly to spiritual training, to living the "holy life" (brahmacariya). For conduct to be completely purified, for sustained contemplation and penetrating wisdom to unfold without impediments, adoption of a different style of life becomes imperative, one which minimizes distractions and stimulants to craving and orders all activities around the aim of liberation. Thus the Buddha established the Sangha, the order of monks and nuns, as the special field for those ready to dedicate their lives to the practice of his path, and in the Dhammapada the call to the monastic life resounds throughout.

The entry-way to the monastic life is an act of radical renunciation. The thoughtful, who have seen the transience and hidden misery of worldly life, break the ties of family and social bonds, abandon their homes and mundane pleasures, and enter upon the state of homelessness (83, 87-89, 91). Withdrawn to silent and secluded places, they seek out the company of wise instructors, and guided by the rules of the monastic training, devote their energies to a life of meditation. Content with the simplest material requisites, moderate in eating, restrained in their senses, they stir up their energy, abide in constant mindfulness and still the restless waves of thoughts (185, 375). With the mind made clear and steady, they learn to contemplate the arising and falling away of all formations, and experience thereby "a delight that transcends all human delights," a joy and happiness that anticipates the bliss of the Deathless (373-374). The life of meditative contemplation reaches its peak in the development of insight (vipassana), and the Dhammapada enunciates the principles to be discerned by insight-wisdom: that all conditioned things are impermanent, that they are all unsatisfactory, that there is no self or truly existent ego entity to be found in anything whatsoever (277-279). When these truths are penetrated by direct experience, the craving, ignorance and related mental fetters maintaining bondage break asunder, and the disciple rises through successive stages of realization to the full attainment of Nibbana.

\item The fourth level of teaching in the Dhammapada provides no new disclosure of doctrine or practice, but an acclamation and exaltation of those who have reached the goal. In the Pali canon the stages of definite attainment along the way to Nibbana are enumerated as four. At the first, called "stream-entry" (sotapatti), the disciple gains his first glimpse of "the Deathless" and enters irreversibly upon the path to liberation, bound to reach the goal in seven lives at most. This achievement alone, the Dhammapada declares, is greater than lordship over all the worlds (178). Following stream-entry come two further stages which weaken and eradicate still more defilements and bring the goal increasingly closer to view. One is called the stage of once-returner (sakadagami), when the disciple will return to the human world at most only one more time; the other the stage of non-returner (anagami), when he will never come back to human existence but will take rebirth in a celestial plane, bound to win final deliverance there. The fourth and final stage is that of the arahant, the Perfected One, the fully accomplished sage who has completed the development of the path, eradicated all defilements and freed himself from bondage to the cycle of rebirths. This is the ideal figure of early Buddhism and the supreme hero of the Dhammapada. Extolled in Chapter 7 under his own name and in Chapter 26 (385-388, 396-423) under the name brahmana, "holy man," the arahant serves as a living demonstration of the truth of the Dhamma. Bearing his last body, perfectly at peace, he is the inspiring model who shows in his own person that it is possible to free oneself from the stains of greed, hatred and delusion, to rise above suffering, to win Nibbana in this very life.

The arahant ideal reaches its optimal exemplification in the Buddha, the promulgator and master of the entire teaching. It was the Buddha who, without any aid or guidance, rediscovered the ancient path to deliverance and taught it to countless others. His arising in the world provides the precious opportunity to hear and practice the excellent Dhamma (182, 194). He is the giver and shower of refuge (190-192), the Supreme Teacher who depends on nothing but his own self-evolved wisdom (353). Born a man, the Buddha always remains essentially human, yet his attainment of Perfect Enlightenment elevates him to a level far surpassing that of common humanity. All our familiar concepts and modes of knowing fail to circumscribe his nature: he is trackless, of limitless range, free from all worldliness, the conqueror of all, the knower of all, untainted by the world (179, 180, 353).

Always shining in the splendor of his wisdom, the Buddha by his very being, confirms the Buddhist faith in human perfectibility and consummates the Dhammapada's picture of man perfected, the arahant.

\end{enumerate}

The four levels of teaching just discussed give us the key for sorting out the Dhammapada's diverse utterances on Buddhist doctrine and for discerning the intention behind its words of practical counsel. Interlaced with the verses specific to these four main levels, there runs throughout the work a large number of verses not tied to any single level but applicable to all alike. Taken together, these delineate for us the basic world view of early Buddhism. The most arresting feature of this view is its stress on process rather than persistence as the defining mark of actuality. The universe is in flux, a boundless river of incessant becoming sweeping everything along; dust motes and mountains, gods and men and animals, world system after world system without number — all are engulfed by the irrepressible current. There is no creator of this process, no providential deity behind the scenes steering all things to some great and glorious end. The cosmos is beginningless, and in its movement from phase to phase it is governed only by the impersonal, implacable law of arising, change, and passing away.

However, the focus of the Dhammapada is not on the outer cosmos, but on the human world, upon man with his yearning and his suffering, his immense complexity, his striving and movement towards transcendence. The starting point is the human condition as given, and fundamental to the picture that emerges is the inescapable duality of human life, the dichotomies which taunt and challenge man at every turn. Seeking happiness, afraid of pain, loss and death, man walks the delicate balance between good and evil, purity and defilement, progress and decline. His actions are strung out between these moral antipodes, and because he cannot evade the necessity to choose, he must bear the full responsibility for his decisions. Man's moral freedom is a reason for both dread and jubilation, for by means of his choices he determines his own individual destiny, not only through one life, but through the numerous lives to be turned up by the rolling wheel of samsara. If he chooses wrongly he can sink to the lowest depths of degradation, if he chooses rightly he can make himself worthy even of the homage of the gods. The paths to all destinations branch out from the present, from the ineluctable immediate occasion of conscious choice and action.

The recognition of duality extends beyond the limits of conditioned existence to include the antithetical poles of the conditioned and the unconditioned, samsara and Nibbana, the "near shore" and the "far shore." The Buddha appears in the world as the Great Liberator who shows man the way to break free from the one and arrive at the other, where alone true safety is to be found. But all he can do is indicate the path; the work of treading it lies in the hands of the disciple. The Dhammapada again and again sounds this challenge to human freedom: man is the maker and master of himself, the protector or destroyer of himself, the savior of himself (160, 165, 380). In the end he must choose between the way that leads back into the world, to the round of becoming, and the way that leads out of the world, to Nibbana. And though this last course is extremely difficult and demanding, the voice of the Buddha speaks words of assurance confirming that it can be done, that it lies within man's power to overcome all barriers and to triumph even over death itself.

The pivotal role in achieving progress in all spheres, the Dhammapada declares, is played by the mind. In contrast to the Bible, which opens with an account of God's creation of the world, the Dhammapada begins with an unequivocal assertion that mind is the forerunner of all that we are, the maker of our character, the creator of our destiny. The entire discipline of the Buddha, from basic morality to the highest levels of meditation, hinges upon training the mind. A wrongly directed mind brings greater harm than any enemy, a rightly directed mind brings greater good than any other relative or friend (42, 43). The mind is unruly, fickle, difficult to subdue, but by effort, mindfulness and unflagging self-discipline, one can master its vagrant tendencies, escape the torrents of the passions and find "an island which no flood can overwhelm" (25). The one who conquers himself, the victor over his own mind, achieves a conquest which can never be undone, a victory greater than that of the mightiest warriors (103-105).

What is needed most urgently to train and subdue the mind is a quality called heedfulness (appamada). Heedfulness combines critical self awareness and unremitting energy in a process of keeping the mind under constant observation to detect and expel the defiling impulses whenever they seek an opportunity to surface. In a world where man has no savior but himself, and where the means to his deliverance lies in mental purification, heedfulness becomes the crucial factor for ensuring that the aspirant keeps to the straight path of training without deviating due to the seductive allurements of sense pleasures or the stagnating influences of laziness and complacency. Heedfulness, the Buddha declares, is the path to the Deathless; heedlessness, the path to death. The wise who understand this distinction abide in heedfulness and experience Nibbana, "the incomparable freedom from bondage" (21-23).

As a great religious classic and the chief spiritual testament of early Buddhism, the Dhammapada cannot be gauged in its true value by a single reading, even if that reading is done carefully and reverentially. It yields its riches only through repeated study, sustained reflection, and most importantly, through the application of its principles to daily life. Thence it might be suggested to the reader in search of spiritual guidance that the Dhammapada be used as a manual for contemplation. After his initial reading, he would do well to read several verses or even a whole chapter every day, slowly and carefully, relishing the words. He should reflect on the meaning of each verse deeply and thoroughly, investigate its relevance to his life, and apply it as a guide to conduct. If this is done repeatedly, with patience and perseverance, it is certain that the Dhammapada will confer upon his life a new meaning and sense of purpose. Infusing him with hope and inspiration, gradually it will lead him to discover a freedom and happiness far greater than anything the world can offer.

— Bhikkhu Bodhi
